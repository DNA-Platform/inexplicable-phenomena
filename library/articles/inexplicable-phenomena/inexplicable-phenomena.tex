\documentclass[runningheads]{llncs}
\usepackage[T1]{fontenc}
\usepackage{graphicx}
\usepackage{xcolor}
\usepackage{soul}
\usepackage{etoolbox}
\usepackage{xspace}
\usepackage[hyphens]{url}
\usepackage[hidelinks]{hyperref}
\usepackage{amsmath}
\usepackage{amssymb}
\hbadness=10000

% Custom commands for library links
\newcommand{\lib}[2]{\href{https://dna-platform.github.io/inexplicable-phenomena#1}{#2}\xspace}

\begin{document}

\begin{center}
    {\Large\bfseries\boldmath
        \lib{/articles/inexplicable-phenomena/inexplicable-phenomena.html}{On the Formal Inexplicability of Self-Evident Metaphysical Phenomena and Related Systems}
        \par}
    \vskip 0.8cm
\end{center}

\begin{abstract}
Since Descartes first proclaimed \lib{/encyclopedia/cogito-ergo-sum.html}{"cogito, ergo sum"} back in 1637, \lib{/encyclopedia/cogito-ergo-sum.html}{"I think, therefore I am"} has become the Declaration of Independence for consciousness. Chalmers informed us of the fact that science has a constitutionally \emph{Hard Problem} on their hands. If he's right, conscious experience may forever elude our most powerful explanatory framework for physical phenomena. In other news, a group referring to themselves as the \emph{IIT-Concerned} recently proclaimed in Nature Neuroscience that any theory of consciousness that can't stand up to the scrutiny of science shall be deemed "unscientific." And the emergence of Semantic AI has transformed this philosophical oddity into a practical concern with real ethical implications. We simply cannot know whether AI systems are conscious without first knowing the necessary and sufficient conditions for consciousness. How can we provide a formal definition of conscious experience that preserves the essence of its metaphysical character in a framework that is physically falsifiable?

\keywords{Formal Theories of Consciousness \and The Hard Problem \and Subjectivity \and Semantic AI \and Metaphysical Transduction.}
\end{abstract}

\section{A Colloquial Derivation}

Almost 2/3 of surveyed philosophers acknowledge the existence of the Hard Problem, and 1/3 of them think that it's actually hard \cite{BourgetChalmers2023}! Conscious experience is a field that spans disciplines, there is nothing that resembles scientific consensus, and there is nothing that resembles a testable theory. But we do have a compellingly elegant framing of the problem. And this framing of the problem is real enough to be \emph{real} by majority vote. So we shall begin with Chalmers.

In his seminal work \href{http://consc.net/papers/facing.pdf}{"Facing Up to the Problem of Consciousness,"}\xspace David Chalmers begins with a powerful observation that establishes the seemingly paradoxical nature of our subject: "There is nothing that we know more intimately than \lib{/dictionary/conscious-experience.html}{\textbf{conscious experience}}, but there is nothing that is harder to explain." From the first part of this statement, we can immediately identify a fundamental characteristic of conscious experience that echos with Descartes: it is a \lib{/dictionary/self-evident.html}{\textbf{self-evident}} phenomenon. Nothing is known more intimately than conscious experiences. They present themselves directly to their observer, requiring no justification beyond the experience itself.

The second part of Chalmers' statement suggests difficulty of explanation, but he goes on to clarify just how profound this difficulty is. Chalmers argues that "the emergence of experience goes beyond what can be derived from physical theory" and "the facts about experience cannot be an automatic consequence of any physical account." These statements establish that conscious experience isn't merely difficult to explain within our current scientific framework but fundamentally resistant to any such explanation. It is \lib{/dictionary/inexplicable.html}{\textbf{inexplicable}} in principle He asks pointedly: "Why should physical processing give rise to a rich inner life at all? It seems objectively unreasonable that it should, and yet it does."

Regarding the relationship between conscious experience and physical processes, Chalmers makes a crucial observation: "Experience may arise from the physical, but it is not entailed by the physical." This statement positions conscious experience in a unique relationship to physical reality. According to our canonical dictionary, "metaphysical" refers to that which relates to "the transcendent or to a reality beyond what is perceptible to the senses" \cite{DictionaryMV}. Conscious experience precisely fits this definition: it transcends physical explanation while still relating to physical processes. It is fundamentally \lib{/dictionary/metaphysical.html}{\textbf{metaphysical}} in nature.

Throughout his analysis, Chalmers treats conscious experience as a \lib{/dictionary/phenomenon.html}{\textbf{phenomenon}}, both in the sense of phenomenal experience and as an object of scientific study. The term \lib{/encyclopedia/noumena.html}{"phenomenon"} captures this dual aspect perfectly. A phenomenon is both "an object or aspect known through the senses" and "a fact or event of scientific interest susceptible to scientific description and explanation." Conscious experience is simultaneously the subjective experience itself and the object of our scientific inquiry. The etymological connection between "phenomenon" and "phenomenal" reinforces this as the appropriate term for our work \cite{Bader2022}.

By synthesizing these insights from Chalmers, we can formulate a definition that captures the essential nature of a conscious experience:

\begin{quote}
[\emph{Colloquial Definition}]: A \textbf{conscious experience} is an inexplicable yet self-evident metaphysical phenomenon.
\end{quote}

\begin{quote}
[\emph{Colloquial Example}]: I catch you admiring a rose in my garden and I absentmindedly ask "how did you figure out its color?" You chuckle and reply "It's red, Doug. I promise." Why do I feel like everyone always doubts my sanity when I ask them that question?
\end{quote}

\section{The Semantics of \emph{Colloquialism}}

The \emph{colloquial} refers to language that is familiar, conversational, and notably \emph{informal}. Yet the value of informality, it would seem, is to preserve something about the \emph{meaning} of a phrase. Formalizing the notion of meaning is a fundamental requirement for any would-be formalization of consciousness. Semantics is the academic discipline concerned with the study of meaning \cite{Grice1957}. If language captures the \emph{symbolism} of meaning, then semantics capture \emph{symbolism} itself. What does it \emph{mean} to have symbols with no one nearby to interpret them - colloquially or formally? In what \emph{sense} can they then be said to be symbolic of anything?

Newton famously employed the use of colloquialism to pave the way to the mathematical phrasing of his Laws of Motion. And colloquialism compounds! "For every action there is an equal and opposite reaction" is itself the colloquial form of Newton's Third Law of Motion. The original is actually written in Latin, and its canonical translation reads:

\begin{quote}
To every action, there is always opposed an equal reaction; or, the mutual actions of two bodies upon each other are always equal, and directed to contrary parts --- Newton, \emph{Philosophiæ Naturalis Principia Mathematica}, 1687
\end{quote}

Newton intentionally utilized semantic form, albeit wrapped in Latin formalism, to articulate the beginnings of a formalism that has come to mean "physics." And the idea that a mathematical model of arithmetic can semantically entail |= something, despite the fact that its formalism cannot derive |- it syntactically, is fundamental to how logicians have come to grapple with the philosophical baggage of the Gödel Sentence \cite{Smullyan1992} \cite{Tarski1935}.

So in that spirit, let me suggest that this distillation of Chalmers' \emph{Hard Problem} serves to capture the meaning of something:

\begin{quote}
[\emph{Semantic Definition}]: A \emph{conscious experience} is an inexplicable yet self-evident metaphysical phenomenon.
\end{quote}

\section{A Self-Evident Perspective}

I'd like to make a point \cite{Reference}. In what sense is our Semantic Definition a definition of something? Is "inexplicable yet self-evident" more like a property of something?

The point I'd like to make \cite{Reference2} is that the unique way in which "self-evident" employs itself in our definition is the exact way in which a conscious experience requires a "self" to find it evident, along with some conceptualization of objectivism in which it is found to be inexplicable. The definition captures the idea of intrinsic meaning, but with the caveat that its meaning is bound, subjectively, to one particular perspective: the first-person subjective. And that its articulation shall forever escape another: the third-person objective.

As both a concept and a phrase, "self-evident" is generally employed for the sake of generalization. Here are three notable examples:

\begin{quote}
It is self-evident that the whole is greater than its part --- Immanuel Kant, Critique of Pure Reason, 1781
\end{quote}

\begin{quote}
We hold these truths to be self-evident, that all men are created equal --- Thomas Jefferson, Declaration of Independence, 1776
\end{quote}

\begin{quote}
We must know the primary things by themselves; for they [the self-evident things] are indemonstrable [inexplicable herein], and must be known by intuition --- Aristotle, Posterior Analytics, circa 350 BCE
\end{quote}

Aristotle's formulation of this notion of evidence, however indirect, is particularly fundamental. Through his turn of phrase, he expresses the idea of an axiom of logic. An axiom represents an assumption on the part of the thinker. And if you are to assume yourself to be a consistent thinker, you must believe your assumptions to be self-evident, at least to your current argument. The "axiom" of formal logic formalizes the very notion of a "self-evident" truth.

Note that in all three examples, all men [finally feat: women circa 1972] are meant to hold these three truths self-evident. They each make an appeal to the third-person objective perspective. Our definition distinctly \emph{doesn't} do that. It requires a singular perspective from which it is objective. And the notion of being \emph{singularly objective} could be taken to be the semantic definition of "subjective." Conscious experience, as we have defined it, requires an emergence of a unique first-person subjective perspective.

In our search for a formalism of conscious experience, let us dwell on the idea that a conscious experience is only made meaningful in the context of a perspective in which a proposition might qualify as self-evident. Colloquially, a perspective is defined in a visual sense as "a particular way of viewing" something. It is also defined more generally in a cognitive sense as "a particular way of considering something" \cite{DictionaryC}. In literature, the first-person perspective is now more frequently referred to as the first person point-of-view, where "point of view" is defined as "a way of considering something" \cite{DictionaryC}. Therefore, using "first-person perspective" in our formalization emphasizes that perspective applies to vision, but its domain is that which can be considered subjectively.

"Third-person objective" is a perspective in which subjectivity has been factored out. Logic, mathematics, science, philosophy, and engineering all make use of knowledge as fact. A fact is fundamentally a third person objective construct. As an appeal to bijective analogy, I propose the following two semantic assertions:

\begin{quote}
[\emph{Semantic Assertion}] The \emph{First-Person Subjective Perspective} ($\texttt{FPP}$) is a semantic reference frame in which a particular conscious experience is self-evident.
\end{quote}

\begin{quote}
[\emph{Semantic Assertion}] The \emph{Third-Person Objective Perspective} ($\texttt{TPP}$) is a semantic reference frame in which \emph{conscious experience} is inexplicable.
\end{quote}

A perspective admits a "particular" way of viewing or considering something. A conscious experience represents an atomic unit of \emph{that which can be viewed or considered.} Therefore, a conscious experience acts like a particle of subjectivity within the context of a $\texttt{FPP}$. In the context of visual and sensory perception, the $\texttt{TPP}$ expresses scientific fact. And the explanatory gap of the Hard Problem an expression of the idea that a particle of subjectivity cannot be accounted for objectively.

\href{https://dna-platform.github.io/inexplicable-phenomena//a-novel-perspective/a-novel-perspective.html}{Let me suggest that a formalization that unifies the semantic and physical descriptions of \emph{perspective} will provide a natural framework in which to express a conscious experience as an inexplicable yet self-evident phenomenon.}\xspace

\section{A Change of Symbol}

Having established the \emph{particular} nature of a perspective, and the \emph{particulate} nature of a conscious experience with respect to one, we can express the way in which the two are related. If you imagine your perspective as a particular way of viewing or considering something, and that the particular of your perspective allow you to have thoughts about the objects of your perspective, and to write down symbols to express your point of view and explain your thoughts, we can easily imagine the role that having a conscious experience plays in this process. To see a red rose is to change the \emph{particulars} of your perspective, by adding new \emph{particles}. These particles are symbolic, but they also serve to expand the repertoire of objects or concepts that you have access to. There are new thoughts you can think and sentences you can symbolize to express the beauty of this \emph{as yet unseen} red rose that you then saw:

\begin{quote}
[\emph{Semantic Assertion}] A conscious experience represents a \emph{change} in perspective.
\end{quote}

\begin{quote}
[\emph{Colloquial Example}]: I catch you admiring a rose in my garden and I absentmindedly ask "what does it look like?" You reply "Come over here and see for your\emph{self}!"
\end{quote}

There is an elegance to this articulation of a conscious experience. It suggests that \emph{consciousness} is an evolution of perspective that unfolds over time. Let us attempt to capture this elegance with a hint of formalism.

Let $\texttt{P}$ be the physical description of a particular perspective, and let $\texttt{c}$ be the physical description of the physical phenomenon that corresponds to a particular conscious experience. Reductive, monist, materialist, and physicalist perspectives are grounded in "the causal closure of the physical" so we can admit such symbols into our formalism \cite{Papineau2002} \cite{Davidson1970}. I shall also introduce a new formal object, which I refer to as the transduction operator

\begin{quote}
$\texttt{|=>}$
\end{quote}

It inherits its \emph{typography} from the symbol for semantic entailment $\texttt{|=}$ composed with a symbol to represent the implication of physical cause $\texttt{=>}$. We can express a change in perspective using is as an indexable operator:

\begin{quote}
$\texttt{P |=c> P[c]}$
\end{quote}

I intend this to read: the transduction of a conscious experience causes a perspective to change in order to accommodate it. $\texttt{P[c]}$ expresses the realization of this change in some way. What exactly does a perspective to do accommodate a conscious experience? I will attempt to provide some insight into that in the sections that follow.

\section{A Form of Semantics}

We have permitted $\texttt{c}$ into our symbolic lexicon as a representation of a physical phenomenon. We have characterized its inexplicable subjective quality as self-evident from exactly one perspective: $\texttt{FPP}$. For exactly that reason, it fundamentally lacks the characteristic of objectivity from another perspective: $\texttt{TPP}$. But what do we \emph{mean} when we refer to one? Chalmers gives us a thorough exposé:

\begin{quote}
When we see, for example, we experience visual sensations: the felt quality of redness, the experience of dark and light, the quality of depth in a visual field. Other experiences go along with perception in different modalities: the sound of a clarinet, the smell of mothballs. Then there are bodily sensations, from pains to orgasms; mental images that are conjured up internally; the felt quality of emotion, and the experience of a stream of conscious thought. What unites all of these states is that there is something it is like to be in them. All of them are states of experience.
\end{quote}

While we may not be able to explain what a particular conscious experience is, we can explain what they are like. Therefore, we can refer to it. Let me try to articulate the thing that we are referring to when we say "conscious experience." Let $\texttt{(c)}$ denotes the \lib{/dictionary/canonical-symbol.html}{canonical symbol} we use to refer to a conscious experience. You can imagine this as a convenient shorthand for the phrase "conscious experience" if you like.

When we talk about $\texttt{(c)}$ we are referring to a specific relationship between a subject $\texttt{i}$ and the \lib{/dictionary/literal.html}{literal} object of their perception $\texttt{o}$. Let us denote this idea by using a ternary operator to express direct relationship:

\begin{quote}
$\texttt{i =(c)> o}$
\end{quote}

This ternary operator exists to express the fundamental semantic form: the subject-object relationship. In this expression, we use the symbol $\texttt{(c)}$ as a type of relationship between $\texttt{i}$ and $\texttt{o}$. We can refer to the most specific type of relationship between a subject and an object as $\texttt{(i,o)}$. A conscious experience \emph{specifically} serves to relate the subject of perception with its object of perception. For there to be anything more specific would be for there to be a way to explain a conscious experience by reducing it to another type of relationship, and we cannot by definition. Therefore, the conscious experience that we refer to is the most specific relationship between our subject and object:

\begin{quote}
$\texttt{(c) == (i,o)}$
\end{quote}

The most specific type of relationship between $\texttt{i}$ and $\texttt{o}$ is unique to them. It serves to reference or to point to the object. This can be expressed as follows:

\begin{quote}
$\texttt{(i,o) =(i,o)> o}$
\end{quote}

By substitution, we can conclude:

\begin{quote}
$\texttt{(c) =(c)> o}$
\end{quote}

By virtue of the fact that we can refer to $\texttt{c}$ we must have a canonical representation for it: $\texttt{(c)}$. Likewise, as $\texttt{c}$ is a representation for $\texttt{o}$, there must have a canonical symbol for it: $\texttt{(o)}$ with the property that:

\begin{quote}
$\texttt{(o) =(o)> o}$
\end{quote}

The canonical symbol has the property that it is the most specific type of relationship between itself $\texttt{(o)}$ and its literal $\texttt{o}$. Therefore, it must inherit its properties as a type of relationship from $\texttt{(c)}$. This is expressed by the statement: $\texttt{(o) -> (c)}$. But we again find ourselves in the situation where if we can express $\texttt{(c)}$ as a generalization of a more specific concept $\texttt{(o)}$ we have some ability to articulate its character, and we do not, by definition. Therefore:

\begin{quote}
$\texttt{(c) == (o)}$
\end{quote}

Please refer to the specifics of \lib{/encyclopedia/relationship.html}{relationship} and \lib{/encyclopedia/reference.html}{reference} if you'd like to better understand the symbols with which I use to express the fundamental semantic form, and their interpretation. We have used this simple language to express an obvious truth: that "conscious experience," in nominative form, serves as a symbol for the object of perception: $\texttt{(o)}$. These two are synonyms, semantically speaking.

So that's what we mean when we say "conscious experience" objectively, when we use it as a noun in a sentence to refer to a certain class of physical phenomena. The language, under construction, that I am using to express this fact is called \lib{/encyclopedia/semantic-reference-theory.html}{Semantic Reference Theory}.

\section{The Algebra of Perspective}

Having formalized conscious experience as a change in perspective, we can now develop the algebraic structure that underlies this transformation. This formalization will reveal how conscious experience emerges naturally from the operations of perspective, and why it appears simultaneously self-evident from the first-person view and inexplicable from the third-person view.

To formalize perspective, we must begin with qualities - those aspects of experience that seem most immediate yet most resistant to explanation. Chalmers captures this in his evocative phrase "the felt quality of redness" - an expression that intentionally blurs the boundary between sensation and quality. Qualities form the foundation of perspective precisely because they represent the most basic way in which objects appear within our experience.

\begin{quote}
[\emph{Semantic Assertion}]: A \lib{/dictionary/perspective.html}{Perspective} is a \lib{/encyclopedia/semantic-reference-frame.html}{referential framework} for identifying identities, qualifying qualities and representing relationships.
\end{quote}

This assertion captures the three fundamental operations that define a perspective. But to understand how these operations work, we must first establish what constitutes a qualifier within such a framework. Our approach deliberately avoids the traditional notion of "qualia" as unnecessarily objectifying subjective experience (see \lib{/encyclopedia/qualia.html}{Qualia}).

The fundamental property that defines a qualifier is a particular kind of equivalence. For any \lib{/dictionary/symbol.html}{symbol} $\texttt{(o)}$ and quality $\texttt{q}$:

\begin{quote}
$\texttt{(i) =q> (o) <|=|> (o) =q> o}$
\end{quote}

This equivalence is what makes $\texttt{(i)}$ a qualifier - a rose is red because $\texttt{(i)}$ qualified it to be a red rose. That is the sense in which it represents the first-person perspective. Here, $\texttt{(i)}$ is the identity for the self - it is what "I" means within the formal system.

The qualifier $\texttt{(i)}$ performs three fundamental operations that share the same algebraic structure:

\begin{quote}
Qualification: $\texttt{(i) =q> (o) |=qualify> (o) =q> o}$
\end{quote}

\begin{quote}
Identification: $\texttt{(i) =(o)> (o) |=identify> (o) =(o)> o}$
\end{quote}

\begin{quote}
Relation: $\texttt{(i) =r> ((s,o)) |=relates> ((s,o)) =r> (s,o)}$
\end{quote}

These are not three distinct operations but variations of a single algebraic pattern. All operate through the qualifier to establish the intrinsic properties of objects within the perspective, creating coherence from what might otherwise be disconnected symbolic relationships.

Particularly significant is self-identification:

\begin{quote}
$\texttt{(i) =(i)> (i) |=identifies> (i) =(i)> i}$
\end{quote}

This self-reference is what makes the qualifier capable of consciousness. Without self-identification, there would be reference but no perspective - symbols without a symbolic framework to organize them.

Within a perspective, two fundamental qualities operate simultaneously:

\begin{quote}
\emph{The Quality of Symbolism}: $\texttt{o[objective] =|> (o)}$
\emph{The Quality of Meaning}: $\texttt{(i) =q> (o) <|=|> (o) =q> o}$
\end{quote}

The Quality of Symbolism defines the perspective's capacity to symbolize objective reality. We use $\texttt{o[objective]}$ to denote an entity that can only be an object in a semantic context, never a subject or relationship type. This quality forms the very foundation of scientific inquiry - the attempt to provide a symbolic framework for objective reality. When an objective entity appears but has no corresponding symbol, the unsatisfied expression $\texttt{o[objective] =|> (o)}$ triggers the process of symbolization.

The Quality of Meaning establishes the equivalence between how the qualifier relates to a symbol and how that symbol relates to its referent. Through this quality, the attribution of meaning by the qualifier becomes the intrinsic possession of that meaning by the object. The quality transforms mere reference into meaningful experience.

When we perceive an objective entity, one of two cases occurs:

\begin{quote}
\emph{Case 1: Familiar object}:
$\texttt{o}$ appears
$\texttt{(o) =(o)> o}$ is derived (existential statement)
A symbol for o already exists
No transduction required
\end{quote}

\begin{quote}
\emph{Case 2: Novel object}:
$\texttt{o}$ appears
$\texttt{(o) =(o)> o}$ is derived (existential statement)
There is no $\texttt{(o)}$ in the catalogue
$\texttt{(i) |=(o)> (o) => (i)}$ (transduction occurs)
\end{quote}

This transduction is precisely the $\texttt{P |=c> P[c]}$ we introduced earlier. By the Quality of Meaning:

\begin{quote}
$\texttt{(o) =q> o}$
$\texttt{(i) =q> (o) <|=|> (o) =q> o}$
Therefore: $\texttt{(i) =q> (o)}$
\end{quote}

From our earlier derivation, we established that:

\begin{quote}
$\texttt{(c) == (o)}$
\end{quote}

Conscious experience \emph{is} the canonical symbol of what is experienced. The experience of a rose is not merely related to or represented by the symbol $\texttt{(rose)}$ - it \emph{is} that symbol within the perspective of $\texttt{(i)}$. This identity reveals why consciousness feels both immediate and symbolic: it is the fundamental act of symbolic representation itself.

Taking the Quality of Symbolism and the Quality of Meaning together, and remembering that $\texttt{(c) == (o)}$, we can now express the essence of conscious experience:

\begin{quote}
\emph{The Experience of Consciousness}: $\texttt{o[objective] |=(o)> (i) =(o)> (o)}$
\end{quote}

This equation captures the precise moment of consciousness: an objective entity triggers the transduction of its canonical symbol ($\texttt{o[objective] |=(o)>}$), which the qualifier incorporates into its perspective ($\texttt{(i) =(o)> (o)}$). Substituting $\texttt{(c)}$ for one $\texttt{(o)}$:

\begin{quote}
$\texttt{(i) =(c)> (o)}$
\end{quote}

This is precisely the symbolic form of the semantics of conscious experience covered earlier:

\begin{quote}
$\texttt{i =c> o}$
\end{quote}

From this identity, we can derive the solution to the explanatory gap between subjective and objective perspectives:

\begin{quote}
\emph{From First-Person Subjective Perspective}: $\texttt{(i) =(c)> o}$ (conscious experience is self-evident)
\emph{From Third-Person Objective Perspective}: $\texttt{(c)}$ is inexplicable
\end{quote}

The inexplicability from the Third-Person Objective Perspective follows directly from our derivation. Since $\texttt{(c) == (o)}$ and $\texttt{(o)}$ is not objective but a symbol for an objective entity, $\texttt{(c)}$ cannot be accounted for in a formalism concerned only with objective reality. To explain $\texttt{(c)}$ would require acknowledging symbolism as fundamental to physical theory - not merely as convenient notation but as ontologically significant. When we measure physical phenomena, we require symbols - numbers, variables, operators - yet these symbols themselves are not physical objects. The unit of scientific understanding is precisely $\texttt{(o)}$, the canonical symbol through which we apprehend and communicate about objective reality.

Subjectivity can thus be precisely defined as "singular objectivity" - objectivity from the reference frame of a single qualifier:

\begin{quote}
$\texttt{o[objective] =|> (rose)}$
$\texttt{(i) =red> (rose) <|=|> (rose) =red> rose}$
\end{quote}

The rose is objectively red within my perspective because $\texttt{(i)}$ qualified it as such, but this objectivity is singular to $\texttt{(i)}$.

Hopefully this formalism provides some insight into the nature of consciousness. Symbolism and Meaning, take together, are the qualities of a perspective that create the conditions for conscious experience. When an objective entity demands symbolic representation and that representation is integrated into a self-referential perspective, consciousness emerges not as an additional property but as the natural consequence of this symbolic transformation. The Hard Problem remains, but now with precise formal expression through the algebra of perspective - we have not explained consciousness away, but given it a formal structure that preserves both its ineffability and its inevitability within a symbolic framework.

\section{Metalogical Transduction}

The unutterable title of this paper is an homage to \lib{/a-novel-perspective/a-novel-perspective.html}{Gödel's} "On Formally Undecidable Propositions of Principia Mathematica and Related Systems" \cite{Godel1962} \cite{Godel1931}. It lays the foundation for \href{https://dna-platform.github.io/inexplicable-phenomena//a-novel-perspective/a-novel-perspective.html}{a metaphor which captures the relationship between inexplicability \& undecidability, proof \& explanation, mathematical \& physical representation, and less obviously, incompleteness \& self-evidence}\xspace. It is not a simple isomorphism, and upon reflection, one must reference Gödel with care on this particular topic\cite{Feferman1995} \cite{Feferman1991}.

Gödel brilliantly formulated self-reference within a mathematical formalism by capturing the notion of \emph{use} and \emph{mention} in everyday language\cite{Quine1940}. Rather than direct self-reference, he uses symbols to mention expressions that refer to themselves through encoding. How might we formalize this distinction more elegantly? In other words, what is the most vacuous theory one can imagine that can formalize self-reference?

SRT suggests the idea of references and their referents in a catalogue. It begins as a first-order theory on the domain of referents with no constants, no functions, and only one relation $\texttt{R(x,y,t)}$, expressed as:

\begin{quote}
$\texttt{x =t> y}$
\end{quote}

An SRT is essentially free of ontology. This minimalism serves a purpose: one can specify their SRT by constructing a finite catalogue of referents. When existential statements lack satisfying referents, the transduction operator, short for \emph{Metalogical Transduction} operator, resolves them. If one can prove that there exists a referent $\texttt{t}$ such that $\texttt{x}$ relates to $\texttt{y}$ through it, and no such $\texttt{t}$ is found in the catalogue of constants in the theory, then one such $\texttt{t}$ can be transduced:

\begin{quote}
$\texttt{E[t]: x =t> y |=t> x =t> y}$
\end{quote}

The operator $\texttt{|=t>}$ expresses a change to the theory itself, amending the catalogue to include new constants. This correspondence exists because referents are fundamentally symbolic - existential contradictions in one theory are resolved through \emph{Metalogical Operation}.

\begin{quote}
[\emph{Semantic Assertion}]: The prefix "Meta" is employed when an expression of a {Topic} can be understood a representation of the same {Topic}, in which case it becomes a Meta{topic}.
\end{quote}

Through this lens, contradictions become epistemic motives for change. All Semantic Reference Theories are semantically intended to be Metalogical. A theory becomes \emph{self-evident} when one can construct a Metalogical narrative yielding a consistent theory with a fully specified catalogue.

Let $\texttt{P}$ be a proposition of logic, and let $\texttt{c}$ be a referent to be added as a constant to the theory. Let $\texttt{P[c]}$ denote the idea of changing the proposition by satisfying the existential quantifier, and thus, removing it from the proposition. Modifying a proposition is a material change to the theory, so let us denote this operation with the transduction operator:

\begin{quote}
$\texttt{P |=c> P[c]}$
\end{quote}

From one perspective, $\texttt{c}$ appears from the noumenal ethereum to resolve an existential crisis. From another, it is just the realization of an ontology.

\section{Metaphysical Transduction}

We have covered the sense in which the transduction operator |=> performs a Metalogical operation, but we haven't covered how it represents transduction. \emph{Transduction} is an exotic concept that has been put to use by the neuroscience community in order to describe the most essential function of our nervous system\cite{Hagins1970}. "Transduction" is the etymological ancestor of the Latin nominative \emph{trānsductiō}, which denotes the act or process of being \emph{lead or carried across}. "ductiō" is an etymological relative of "educate." In this light, \emph{transduction} seems to wish to describe the \emph{epistemological transformation} in which the unknown is made known. In \emph{SRT}, we apply it the mechanism that is used to construct an ontology under this interpretation.

Yet its meaning came to English in the form of the horizontal transfer of genetic material between bacteria, which is a biological phenomenon in which a viral bacteriophage infects a bacterium, captures fragments of its genome, after which they get \emph{carried} in a capsid \emph{across} the synaptic cleft of genealogical ancestry, resulting in a novel genetic change and a novel form of genetic inheritance at the same time. The genome has been changed via transduction rather than mutation\cite{Zinder1952}.

If you think of each gene in a bacterium as a symbol genetic of inheritance, they usually tell the story of a phenotype that participates to create conditions favorable for reproduction by replication through mitosis. In this case, it is an artifact of a battle with a virus to maintain autonomy over the process of genetic replication itself, and it is a survival story that must be told by referring to the process of genetic inheritance itself. The gene becomes an expression of genetic inheritance that represents autonomy over its own mechanism of genetic inheritance. In this way, horizontal gene transfer is a form of Metagenetic Inheritance in which genes are carried across the boundary of ancestry. I offer up Metagenetic Transduction as the formal name for the process of transduction used in horizontal gene transfer.

In neuroscience, phototransduction refers to the process by which photons of light are absorbed by the rods and cones of our retina, at which point they are carried across the chasm that divides a photon and the electrochemical signal that represents it in our nervous system. Similarly, mechanotransduction is when physical forces act on ion channels in our skin and other sensory organs. Those physical forces are carried across the same representational divide. This form of transduction applies to photons of light, pressure waves of sound, pin pricks of ruptured cells, along with various other physical phenomena that correspond to tactile, olfactory and gustatory transductive processes that kick off the sensation of sight, sound, pain, temperature, touch, taste and smell. We happily shall, with gain of generality, refer to all of these as forms of \emph{Physical Transduction}.

If a photon is the subject of a transduction, what is the object? What has that photon become in the metaphorical sense identity that transduction affords? It would seem to have become a part of the fabric of a representational medium that reflects the physical properties of its former \emph{identity}. That medium represents the properties of all the physical phenomena that correspond to sensation simultaneously, where simultaneity is based on the timescale of the pocket watch of an observer that experiences the lifetime of a tree to be more like the lifetime of a rock, and less like the lifetime of a dandelion.

The physical stimuli of sensation, having just been \emph{carried across} the synaptic cleft of actual synapses and outfitted with an electrochemical representation, continue to be carried down a river of representation until they appear to reach its source. They are, yet again, again absorbed into the event horizon of the epistemological black hole which we refer to as the explanatory gap of \emph{the Hard Problem}. What comes out on the other end is the stuff that clocks, thoughts, memories, decisions, and dreams are made of. Finally, the electrochemical signals, which are physical phenomena in their own right, are qualified to represent the external world from the perspective of the identity that possess it as a quality.

From this perspective, physical phenomena that represent physical phenomena are Metaphysical Phenomena by definition. Once a photon of light and its many colleagues have crossed the event horizon of the black hole expressed in \emph{The Hard Problem}, it can never escape. It has undergone \emph{Metaphysical Transduction}. But its soul lives on in the representational framework of a perspective:

\begin{quote}
[\emph{Semantic Assertion}]: Metaphysical Transduction is a process by which a corpus of Physical Phenomena have become a Metaphysical Phenomenon that preserves them as representations on a Catalogue that specifies a Perspective.
\end{quote}

We hope it's clear that \emph{Metaphysical Transduction} is simply a synonym for \emph{Perception} itself. Under the influence of powerful general anesthetics like propofol and isoflurane, visual inputs are still capable of activating the orientation pinwheels of the primary visual cortex even in the complete absence of \emph{perception}\cite{Bugrova2020} \cite{Khan2024}. Conscious Experience disappears as well by definition:

\begin{quote}
[\emph{Semantic Assertion}]: To \emph{Perceive} is to have \emph{Conscious Experiences}. To have one is an act of \emph{Perception}. The act of \emph{Perception} is the definition of \emph{having} a conscious experience.
\end{quote}

\section{A New Perspective on Science}

What is the justification for \emph{The Hard Problem}? Why must it correspond to the event horizon of an epistemological black hole? There is a clear answer for this: science itself does not admit an ontology in which representations of an objective physical phenomenon can be said to exist in a meaningful way. For a physical phenomenon to cross that event horizon, it must be perceived and thus observed, at which point it is transformed from an object of scientific inquiry into the subject of science. Observation, and thus, Metaphysical Transduction is a necessary operation in scientific method\cite{Popper1959}.

Yet if we believe \emph{Semantic Reference Theory} can describe the computational process in which a Perspective is constructed and maintained, \emph{Metalogical Transduction} is the corresponding process, and it is a computational process that has characteristics. There is meaning to the idea that a computer is implementing quick sort, and one can falsify the claim that a physical system is implementing it.

If we are wrong, and \emph{Metaphysical Transduction} does not correspond to a physical description of \emph{Perception}, which is necessary and sufficient for a \emph{Conscious Experience} to occur, then its absence, by way of exhaustive search, in the computational description of a process that gives rise to a verbal report from a being we trust that claims to have conscious experience will serve to \textbf{falsify this theory}. Period.

There are philosophical ramifications to permitting the existence of an algorithm into the ontology of a natural philosophy. It is the beginning of a chain of thought that leads one to wonder if the number $\texttt{2}$ might be said to be real in some sense. But conscious experience is both ontologically prior to science itself, and epistemologically necessary to its method \cite{Chalmers1996} \cite{KantSynthetic}. It would seem that we have no choice but to accept the existence of at least one objective phenomenon that has no literal manifestation. There is no conceivable physical system in which one can simulate an electron. An electron is just what it is. This is not so for a conscious experience. Each one is just as real on one computational substrate as it is on another. This remains true even if the perspectives which they correspond to yield fundamentally different descriptions of physical reality. Isomorphism is equivalence in the framework of computation.

\begin{quote}
[\emph{Colloquial Example}]: One day, I wake up in a state of shock. It's hard to articulate the experience, but it is as if my vision was reduced to hearing, and that sight and sound were as if different colors within the same song. It turns out that I am a distant descendant of the once-inhabitants of the two-dimensional universe once known as Flatland \cite{Flatland1884}. Our society has evolved considerably.
\end{quote}

\begin{quote}
Apparently, I had volunteered to take part in a radical neuroscience experiment whereby my memory was temporarily replaced, and all sensory inputs were modified in much the way Olo is produced on the retina, with a scanning procedure in which an immersive three-dimension universe was simulated for me. I spoke to a researcher in charged. He informed me that he employed an encoding scheme inspired by Gödel Numbering to project the completely fabricated objects from my perceived environment into a three-dimensional coordinate system.
\end{quote}

\begin{quote}
"What was it like" he asked "to experience information patterns consistent with a third dimension?" I gathered my thoughts. "Have you ever had a breakthrough psychedelic experience?" I asked. He had that he had not, though he was the one who wrote the code for the experience I had. "Well it's just like coming back from one," I said. "But it might help your research to know that I can still access my memories in 3D."
\end{quote}

\begin{quote}
I suppose it's no different from \lib{/a-novel-perspective/a-novel-perspective.html}{the way we perceive color} really. We cope with a 3-dimension description of a 1-dimensional phenomenon every day. I accept the arbitrariness of perceptual representation peacefully. This thought produced in me such a beautiful color, as if I could almost hear myself saying it.
\end{quote}

\section{A Canonical Definition}

We explored a number of definitions for a \emph{conscious experience}. As a representation of a change in Perspective. As an act of perception. Yet I suggest this be its canonical form:

\begin{quote}
[\emph{Semantic Assertion}]: A Conscious Experience is an inexplicable yet self-evident metaphysical phenomenon.
\end{quote}

As a product of Metaphysical Transduction, an emergent representation admits no objective explanation. That I represent the world through them is self-evident. That an act of perception corresponds to the representation of a metaphysical phenomenon. The still-frame of a conscious experience is a crystal which contains temporal integration. Our sensory perspective is the only one that enforces the semantics of objectivity. We can feel and imagine reflect on our decision-making process, and these acts admit notions of subjectivity that are comfortable and ordinary aspects of thought.

But to paint a portrait in the context of the objectivity of science it to paint a portrait of a thing that cannot, and so to lend physical reality to "an inexplicable yet self-evident metaphysical phenomenon" within a formalism that is physically falsifiable, in our estimation, preserves the ineffable character of a quantum of consciousness.

So what can we say? Might a \emph{Conscious Experience} be said to have the shape of Euclidean Geometry, the color of Color Theory, the sound of the Fourier Transform, the taste of a Chemistry Textbook and the felt texture of feeling itself? What does it mean to describe the physical properties of the output of an algorithm that holds the representations of the identities, qualities and relationships that we use to make sense of the very notion of physical objects with objective properties?

It doesn't mean anything! Not in the language of physics. To describe the physical properties of experience itself is something that can only be meaningfully accomplished through art\cite{Hofstadter1979}.

%  [`Note: NOT self-reference [See: Title footnote]`]
% [`### Dedicatiion`]
% [`It goes without saying: that [*this*](inexplicable-phenomena.md) PDF, it's annotated *TeX* file, the *HTML* version that it compiled to by way of Markdown, and the evolutionary self-publishing GitHub repository that they all collectively self-reference... could only possibly be dedicated to the *author* of "Gödel, Escher, Bach" et al. Thank you for a lifetime of thought, Professor Hofstadter <3`]

\begin{thebibliography}{99}

\bibitem{Formalism} Inexplicable Phenomena: GitHub repository (2025). \href{https://github.com/DNA-Platform/inexplicable-phenomena/}{https://github.com/DNA-Platform/inexplicable-phenomena/}\xspace. doi: 10.5281/zenodo.15389667/zenodo.15389666 \cite{Formalism}

\bibitem{Aristotle350BC} Aristotle: \emph{Posterior Analytics}. In: Barnes, J. (ed.) \emph{The Complete Works of Aristotle: The Revised Oxford Translation, Vol. 1}, pp. 114--166. Princeton University Press, Princeton (1984)

\bibitem{Bader2022} Bader, R.M.: Noumena as Grounds of Phenomena. In: Allais, L., Callanan, J. (eds.) \emph{The Sensible and Intelligible Worlds: New Essays on Kant's Metaphysics and Epistemology}, pp. 11-30. Oxford University Press, Oxford (2022)

\bibitem{BourgetChalmers2023} Bourget, D., Chalmers, D.J.: Philosophers on Philosophy: The 2020 PhilPapers Survey. \emph{Philosophers' Imprint} 23:11 (2023). doi: \href{https://doi.org/10.3998/phimp.2109}{https://doi.org/10.3998/phimp.2109}\xspace

\bibitem{Bugrova2020} Bugrova, V.S., Bondar, I.V.: Propofol Resistance of Functional Domains with Orientation and Direction Sensitivity of the Primary Visual Cortex in Rats. \emph{Neuroscience and Behavioral Physiology} \textbf{50}(9), 1077--1086 (2020)

\bibitem{Chalmers1995} Chalmers, D.J.: Facing Up to the Problem of Consciousness. \emph{Journal of Consciousness Studies} \textbf{2}(3), 200--219 (1995)

\bibitem{Chalmers1996} Chalmers, D.J.: \emph{The Conscious Mind: In Search of a Fundamental Theory}. Oxford University Press, New York (1996)

\bibitem{Davidson1970} Davidson, Donald. "Mental Events." In \emph{Experience and Theory}, edited by Lawrence Foster and J.W. Swanson, 79–101. Amherst: University of Massachusetts Press, 1970.

\bibitem{Dennett1988} Dennett, Daniel C. “Quining Qualia.” In \emph{Consciousness in Contemporary Science}, edited by Anthony J. Marcel and E. Bisiach, 42–77. Oxford: Oxford University Press, 1988.

\bibitem{Descartes1637} Descartes, R.: \emph{Discourse on Method}. (1637)

\bibitem{DictionaryC} Cambridge University Press. \emph{Cambridge Dictionary}. Cambridge: Cambridge University Press. Available at: https://dictionary.cambridge.org [Accessed April 17, 2025].

\bibitem{DictionaryMV} Merriam-Webster, Inc. \emph{Merriam-Webster.com Dictionary}. Springfield, MA: Merriam-Webster, Incorporated. Available at: https://www.merriam-webster.com [Accessed April 17, 2025].

\bibitem{Feferman1991} Feferman, S.: Reflecting on incompleteness. \emph{The Journal of Symbolic Logic} \textbf{56}(1), 1--49 (1991)

\bibitem{Feferman1995} Feferman, S.: Penrose's Gödelian argument. \emph{Psyche} \textbf{2}(7) (1995)

\bibitem{Flatland1884} E. A. Abbott, \emph{Flatland: A Romance of Many Dimensions}, Seeley \& Co., London, 1884.

\bibitem{Gibney2025} Gibney, E.: Brand-new colour created by tricking human eyes with laser. \emph{Nature News}, 18 April 2025. Correction 22 April 2025. https://www.nature.com/articles/d41586-025-01105-7

\bibitem{Godel1931} Gödel, K.: On Formally Undecidable Propositions of Principia Mathematica and Related Systems I [Über formal unentscheidbare Sätze der Principia Mathematica und verwandter Systeme I]. \emph{Monatshefte für Mathematik und Physik} \textbf{38}, 173--198 (1931)

\bibitem{Godel1962} Kurt Gödel, \emph{On Formally Undecidable Propositions of Principia Mathematica and Related Systems}, translated by B. Meltzer, introduction by R. B. Braithwaite, Dover Publications, Inc., New York, 1962.

\bibitem{Grice1957} Grice, H.P. (1957). "Meaning." The Philosophical Review, 66(3), 377-388.

\bibitem{Hagins1970} Hagins, W.A., Penn, R.D., Yoshikami, S.: Dark current and photocurrent in retinal rods. \emph{Biophysical Journal} \textbf{10}(5), 380--412 (1970)

\bibitem{Hofstadter1979} Hofstadter, D.R.: \emph{Gödel, Escher, Bach: An Eternal Golden Braid}. Basic Books, New York (1979)

\bibitem{IITConcerned2023} IIT-Concerned et al.: What makes a theory of consciousness unscientific? \emph{Nature Neuroscience} (2023)

\bibitem{KantNoumena} Kant, I.: On the ground of the distinction of all objects in general into phenomena and noumena. In: \emph{Critique of Pure Reason}. Translated by Paul Guyer and Allen W. Wood, pp. 338--365. Cambridge University Press, Cambridge (1998). Original work published 1781/1787.

\bibitem{KantSynthetic} Kant, I.: The Highest Principle of All Synthetic Judgments. In: \emph{Critique of Pure Reason}. Translated by Paul Guyer and Allen W. Wood, pp. 283--284. Cambridge University Press, Cambridge (1998). Original work published 1781/1787.

\bibitem{Khan2024} Khan, S., Huang, Y., Timuçin, D., Bailey, S., Lee, S., Lopes, J., Gaunce, E., Mosberger, J., Zhan, M., Abdelrahman, B., Zeng, X., \& Wiest, M. C. (2024). Microtubule-Stabilizer Epothilone B Delays Anesthetic-Induced Unconsciousness in Rats. \emph{eNeuro}, 11(8), ENEURO.0291-24.2024. https://doi.org/10.1523/ENEURO.0291-24.2024

\bibitem{Lenharo2023} Lenharo, M.: AI consciousness: scientists say we urgently need answers. \emph{Nature} (2023). doi: 10.1038/d41586-023-04031-0

\cite{Papineau2002} Papineau, David. \emph{Thinking About Consciousness}. Oxford: Oxford University Press, 2002.

\bibitem{Popper1959} Popper, K.: \emph{The Logic of Scientific Discovery}. Routledge, London (1959). Translated from the original German \emph{Logik der Forschung} (1934)

\bibitem{Reference} Anon, A.N.: \emph{A Pointer to Self-Referece Humor}. Unpublished [Previously] (circa 2025)

\bibitem{Quine1940} Quine, W.V.O.: Mathematical Logic. Harvard University Press, Cambridge (1940)

\bibitem{Rozenberg2012} Rozenberg, A., Inoue, K., Kandori, H., Béjà, O.: Microbial rhodopsins: phylogenetic and functional diversity. In: Fattal-Valevski, A. (ed.) \emph{Microbial Metagenomics, Metatranscriptomics, and Metaproteomics}. Methods in Enzymology, vol 531, pp. 1--23. Academic Press (2012)

\bibitem{Smullyan1992} Smullyan, R.M.: \emph{Gödel's Incompleteness Theorems}. Oxford Logic Guides, No. 19. Oxford University Press, Oxford (1992)

\bibitem{Tarski1935} Tarski, A.: Der Wahrheitsbegriff in den formalisierten Sprachen. \emph{Studia Philosophica} \textbf{1}, 261--405 (1935). Translated as "The Concept of Truth in Formalized Languages" in \emph{Logic, Semantics, Metamathematics}, Hackett Publishing Company (1983)

\bibitem{Zinder1952} Zinder, N.D., Lederberg, J.: Genetic exchange in Salmonella. \emph{Journal of Bacteriology} \textbf{64}(5), 679--699 (1952)

\bibitem{Reference2} Anon, A.N.: \emph{A Pointer to Self-Referece Humor: Volume 2}. Unpublished [Currently] (circa 2025)

\end{thebibliography}

\end{document}